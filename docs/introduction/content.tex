%----------------------------------------------------------------------------------------
%	INTRODUCTION
%----------------------------------------------------------------------------------------

\section{Einleitung}

\subsection{Nutzungsproblem}

Im Unternehmen \emph{antwerpes ag} wird, zunehmend auch teamübergreifend, mit REST APIs gearbeitet. Applikationen bieten Schnittstellen an, die von Entwicklern, teilweise in anderen Teams, konsumiert werden. Dabei entstehen typische Kollaborationsprobleme: nicht ausreichende bzw. schlechte Dokumentation und Inkonsistenzen in den Definitionen der Schnittstellen, da sich diese insbesondere während der Entwicklungszeit häufig ändern.\\
Gleichzeitig werden dabei viele Aufgaben der Entwicklung und Interaktion mit den APIs (bspw. Validierung von Anfragen, Testing der Endpunkte oder Konfiguration von Tools wie Postman) vernachlässigt oder manuell ausgeführt.\\
Dies führt im besten Fall zu mehr Zeitaufwand und Code-Duplizierung, im schlimmsten Fall zu weiteren Inkonsistenzen, welche dann manuell korrigiert werden müssen.

\subsection{Zielsetzung}

Unterstützend für eine gute Entwicklung ist ein Entwicklungsprozess, der das Projekt in allen Punkten, die einheitlich organisiert und durchgeführt werden sollten, festlegt; sowie Techniken verwendet, welche die Tätigkeiten möglichst gut unterstützen \cite{ludewig2007software}. Es soll daher ein System entworfen und entwickelt werden, welches ein einheitliches Konzept zum Arbeiten mit APIs einführt. \\
Ziel des Projektes ist die Optimierung des Design- und Entwicklungsprozesses von auf HTTP basierenden REST APIs durch Verwendung einer API Spezifikation als \emph{single source of truth} für Dokumentation und Testing. Dadurch können einige manuelle Aufgaben automatisiert werden, und viele Inkonsistenzen behoben werden. Insgesamt steigt die Testabdeckung und Qualität der Dokumentation der Schnittstellen, womit die Kollaboration zwischen Entwicklern verbessert wird. Das resultierende System soll dabei mit möglichst geringem Aufwand in bereits bestehende Projekte integriert werden können. \\
 