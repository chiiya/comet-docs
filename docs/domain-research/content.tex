%----------------------------------------------------------------------------------------
%	INTRODUCTION
%----------------------------------------------------------------------------------------

\section{Domäne}

\subsection{Einleitung}

Representational State Transfer (REST) ist eine Abstraktion der architektonischen Elemente innerhalb eines verteilten Hypermediasystems. Es ist somit ein Softwarearchitekturstil, der die architektonischen Prinzipien, Eigenschaften und Einschränkungen für die Umsetzung von internetbasierten verteilten Systemen definiert \cite{fielding2000architectural}[86]. REST basiert auf fünf Kernprinzipien \cite{tilkov2015rest}[11]:
\begin{itemize}
	\item Ressourcen als Abstraktion von Informationen, identifiziert durch einen eindeutigen \emph{resource identifier}
	\item Verknüpfungen / Hypermedia
	\item Standardmethoden
	\item Unterschiedliche Repräsentationen
	\item Statuslose Kommunikation
\end{itemize}

In Folgendem wird untersucht, welche maschinenlesbare Spezifikationsformate zur Beschreibung von auf HTTP basierenden REST APIs zur Verfügung stehen. Ziel dieser Untersuchung ist es, eine abstrakte Modellierung von REST APIs zu konzipieren, die verwendet werden kann, um eine API Spezifikation in eine interne Datenstruktur zu überführen, welche dann anschließend angereichert wird um automatisiert Artefakte wie Dokumentation oder Test-Cases zu generieren. Im Zuge dessen wird ebenfalls analysiert, welche Möglichkeiten in diesem Feld bereits existieren.
 