%----------------------------------------------------------------------------------------
%	INTRODUCTION
%----------------------------------------------------------------------------------------

\section{Einführung in die Thematik}

Innerhalb dieses in die Thematik einführenden Kapitels werden Grundlagen bezüglich der Bereiche, in denen sich das Projekt bewegt, aufgezeigt. Zunächst wird ein kurzer Überblick über die grundlegenden Prinzipien von HTTP und REST gegeben. Anschließend wird untersucht, wie APIs mithilfe von verschiedenen Deskriptionsformaten maschinenlesbar beschrieben werden können. Ziel dieser Untersuchung ist es, eine Grundlage für die abstrakte Modellierung von REST APIs zu schaffen, die verwendet werden kann, um eine API Spezifikation in eine interne Datenstruktur zu überführen, welche dann anschließend angereichert wird um automatisiert Artefakte wie Dokumentation und Test-Cases zu generieren. Im Zuge dessen wird im letzten Abschnitt analysiert, welche Möglichkeiten in diesem Feld bereits existieren.

\subsection{HTTP}

Das \emph{Hypertext Transfer Protocol} (HTTP) ist ein Anwendungsprotokoll zur Kommunikation über ein Netzwerk. HTTP ist das primäre Kommunikationsprotokoll im World Wide Web, und wurde von der Internet Engineering Task Force (IETF) und dem World Wide Web Consortium (W3C) in einer Reihe von RFC-Dokumenten (z.B. \href{https://tools.ietf.org/html/rfc7230}{RFC 7230}, \href{https://tools.ietf.org/html/rfc7231}{RFC 7231} und \href{https://tools.ietf.org/html/rfc7540}{RFC 7540}) standardisiert. In diesem Abschnitt werden die wichtigsten Charakteristiken aus den aufgeführten RFCs zusammengefasst. \\

Eine HTTP-Nachricht wird in der Regel über TCP übertragen, und besteht aus 4 Komponenten:

\begin{description}
	\item Verb / Methode: Operationstyp, der durchgeführt werden soll, beispielsweise das Anfragen einer Ressource.
    \item Ressourcenpfad: Ein Bezeichner, der angibt, auf welche Ressource die HTTP-Operation angewendet werden soll.
    \item Headers: Zusätzliche Metadaten, ausgedrückt als Liste von Key/Value-Paaren.
    \item Body: Enhält die Nutzdaten der Nachricht.
\end{description}

Das HTTP Protokoll erlaubt die folgenden Operationen bzw. Methoden:

\begin{description}
	\item \emph{GET}: Die angegebene Ressource soll im Body-Teil der Antwort zurückgegeben werden.
	\item \emph{HEAD}: Wie \emph{GET}, nur dass die Nutzdaten nicht zurückgegeben werden sollen. Dies ist nützlich, wenn nur überprüft wird, ob eine Ressource existiert, oder wenn nur die Header abgerufen werden sollen.
	\item \emph{POST}: Daten werden an den Server geschickt. Häufig wird diese Methode benutzt, um neue Ressourcen anzulegen.
	\item \emph{DELETE}: Die angegebene Ressource löschen.
	\item \emph{PUT}: Die angegebene Ressource mit neuen Daten ersetzen.
	\item \emph{PATCH}: Ändert eine Ressource, ohne diese vollständig zu ersetzen.
	\item \emph{TRACE}: Liefert die Anfrage so zurück, wie der Server sie empfangen hat.
	\item \emph{OPTIONS}: Liefert eine Liste der vom Server unterstützten Methoden auf einer Ressource.
	\item \emph{CONNECT}: Eine Tunneling-Verbindung über einen HTTP-Proxy herstellen, die normalerweise für verschlüsselte Kommunikationen benötigt wird.
\end{description}

Wenn ein Client eine HTTP-Anfrage sendet, schickt der Server eine HTTP-Antwort mit Headern und möglicherweise Nutzdaten im Body zurück. Darüber hinaus enthält die Antwort auch einen numerischen, dreistelligen Statuscode. Es gibt fünf Gruppen von Statuscodes, die durch die erste Ziffer identifiziert werden können:

\begin{description}
	\item \emph{1xx}: Wird für informierende Antworten verwendet, wie z.B. der Meldung, dass die Bearbeitung der Anfrage trotz der Rückmeldung noch andauert.
	\item \emph{2xx}: Wird zurückgegeben, wenn die Anfrage erfolgreich bearbeitet wurde. Der Server könnte beispielsweise weiter spezifizieren, dass eine neue Ressource angelegt wurde (201), z.B. durch einen POST-Befehl, oder dass im Antwortkörper nichts erwartet wird (204), z.B. durch einen DELETE-Befehl.
	\item \emph{3xx}: Wird für Umleitungen benutzt. So kann dem Client bspw. mitgeteilt werden, dass sich eine Ressource an einem neuen Ort befindet.
	\item \emph{4xx}: Bei der Bearbeitung der Anfrage ist ein Fehler aufgetreten, der durch den Client verursacht wurde. Dies können beispielsweise falsch formatierte Anfragen (400), Anfragen die vom Server nicht bearbeitet werden können (422), oder auch Anfragen auf nicht existierende Ressourcen (404) sein.
	\item \emph{5xx}: Bei der Bearbeitung der Anfrage ist ein Fehler aufgetreten, dessen Ursache beim Server liegt. 
\end{description}


\subsection{REST}

Representational State Transfer ist ein Muster von Ressourcenoperationen, das sich als Industriestandard für das Design von Webanwendungen etabliert hat \parencite[2]{battle2008bridging}. Während der traditionelle SOAP-basierte Ansatz für Web Services vollwertige Remote-Objekte mit Remote-Methodenaufruf und gekapselter Funktionalität verwendet, beschäftigt sich REST nur mit Datenstrukturen und der Übertragung ihres Zustands. \\

Representational State Transfer (REST) ist ein Softwarearchitekturstil, der die architektonischen Prinzipien, Eigenschaften und Einschränkungen für die Umsetzung von internetbasierten verteilten Systemen definiert \parencite[86]{fielding2000architectural}. REST basiert auf fünf Kernprinzipien \parencite[11]{tilkov2015rest}:

\begin{itemize}
	\item Ressourcen als Abstraktion von Informationen, identifiziert durch einen eindeutigen \emph{resource identifier} (die URI).
	\item Verknüpfungen / Hypermedia (HATEOAS). Jede Repräsentation einer Ressource sollte ebenfalls Links zu anderen relevanten Ressourcen enthalten. Da in der Praxis die Verwendung von HATEOAS jedoch recht selten ist \parencite[35--36]{rodriguez2016rest} werden im Folgenden auch solche APIs als RESTful bezeichnet, welche dieses Prinzip nicht umsetzen.
	\item Verwendung von Standardmethoden. Der Zugriff auf Ressourcen und deren Manipulation erfolgt mit den in HTTP definierten Standardmethoden. Jede Methode hat ihr eigenes standardisiertes Verhalten und erwartete Statuscodes.
	\item Unterschiedliche Repräsentationen. Clients kennen nicht direkt das interne Format und den Zustand der Ressourcen; sie arbeiten mit Ressourcendarstellungen (z.B. JSON oder XML), die den aktuellen oder beabsichtigten Zustand einer Ressource darstellen. Die Deklaration von Inhaltstypen in den Headern von HTTP-Nachrichten ermöglicht es Clients und Servern, Darstellungen korrekt zu verarbeiten \parencite[23]{rodriguez2016rest}.
	\item Statuslose Kommunikation. Interaktionen zwischen einem Client und einer API sind zustandslos, d.h. jede Anfrage enthält alle notwendigen Informationen, die von der API zur Verarbeitung benötigt werden müssen; es wird kein Zustand auf dem Server gespeichert.
\end{itemize}

\subsection{Spezifikationsformate}

In den letzten Jahren wurde viele Vorschläge gemacht, wie REST APIs menschen- und maschinenlesbar beschrieben werden können. Allen gemeinsam ist, dass die grundlegenden Eigenschaften von REST und HTTP (Ressourcen, Operationen, Requests, Responses, Parameter, usw.) in einem Modell zusammengefasst werden. Hierfür existieren eine Vielzahl von \emph{Description Languages} (DLs), von denen in diesem Projekt OpenAPI, die am weitesten verbreitete \parencite{scherer2016description}, als Basis benutzt wird.

\subsubsection{JSON Schema}

Aufgrund seiner leichten Verständlichkeit für Menschen wie auch Maschinen hat sich JSON zum beliebtesten Format entwickelt, um API-Anfragen und Antworten über das HTTP-Protokoll zu senden \parencite[263]{pezoa2016foundations}. Durch diese Popularität ist allerdings auch der Bedarf gestiegen, mithilfe eines Schemas den Aufbau eines JSON Dokuments konkret zu definieren. Dies kann z.B. für die Validierung von Anfragen oder Antworten dienen. Ohne Integritätsschicht müssen viele Fälle berücksichtigt werden, die bei fehlerhaften API-Aufrufen auftreten, was vermieden werden kann, wenn eine Schemadefinition verwendet wird, um Dokumente herauszufiltern, die nicht die richtige Form haben. \\

JSON Schema ist eine einfache Schemasprache, die es Benutzern ermöglicht, die Struktur von JSON-Dokumenten einzuschränken. Die Definition ist noch lange kein Standard (die Spezifikation befindet sich derzeit im siebten Entwurf), aber es gibt bereits eine wachsende Anzahl von Anwendungen, die JSON-Schemadefinitionen unterstützen, und eine Vielzahl von Tools und Paketen, die die Validierung von Dokumenten anhand von JSON-Schema ermöglichen \parencite[264]{pezoa2016foundations}. Dies kann selbst zur Client-seitigen Validierung von Benutzereingaben verwendet werden \parencite{sturgeon2018validation}. \\

JSON Schema stellt mehrere Schlüsselwörter zur Beschreibung und Validierung von JSON Dokumenten zur Verfügung. Beispielsweise kann so der Typ von Daten festgelegt werden (\lstinline|null|, \lstinline|boolean|, \lstinline|object|, \lstinline|array|, \lstinline|number|, \lstinline|integer| oder \lstinline|string|). Es können ebenfalls vielzählige Restriktionen zur Validierung definiert werden, unter anderem: \lstinline|enum|, \lstinline|minimum|, \lstinline|maximum|, \lstinline|minLength|, \lstinline|maxLength|, \lstinline|regex| (Validierung anhand eines RegEx-Musters), \lstinline|required|, \lstinline|email|, \lstinline|date-time| und \lstinline|uri|. \\

\subsubsection{OpenAPI}

Die OpenAPI DL ist Teil des Swagger Frameworks, einem Open-Source-Framework, das aus einem JSON- bzw. YAML-Format zur Beschreibung von RESTful APIs und aus diversen Werkzeugen, die daraus zum Beispiel eine Dokumentation generieren können, besteht. OpenAPI und JSON Schema sind zwei verschiedene Spezifikationen, die sich gegenseitig inspiriert haben, sich aber subtil unterscheiden. \\

OpenAPI zielt darauf ab, sowohl das Servicemodell zu beschreiben (die API im Allgemeinen, Endpunkte, Anfragemetadaten wie Header, Authentifizierungsstrategien, Antwortmetadaten usw.), als auch den HTTP-Request/Response-Body mit einem Bündel von Schlüsselwörtern basierend auf dem JSON-Schema, die im Laufe der Zeit divergiert haben. Es gibt aktuell Bemühungen, diese Divergenz zu lösen \parencite{sturgeon2018divergence}. JSON Schema wiederum zielt darauf ab, eine Instanz von JSON-Daten zu beschreiben, wie sie in einer HTTP-Anfrage oder -Antwort gefunden wird, ist aber in keiner Weise auf eine HTTP-API beschränkt \parencite{sturgeon2018harmony}. \\

Die OpenAPI DL ist ähnlich wie andere Spezifikationsformate strukturiert: zunächst werden allgemeine Informationen zur API, wie Name, Version, Pfad und Authentifizierungsverfahren festgelegt; im Anschluss folgen die Definitionen der einzelnen Endpunkte, mit ihren Parametern und Requests bzw. Responses.

\subsection{Testing von REST APIs}

In diesem Abschnitt wird ein kurzer Überblick über die wichtigsten Testverfahren gegeben, mit denen REST APIs getestet werden können:

\subsubsection{Funktionstests}
Die wichtigste testbare Eigenschaft eines Programms ist seine Funktionalität, die durch die Anforderungen spezifiziert ist \parencite[457]{ludewig2007software}. Funktionstests sind im Wesentlichen ein Test spezifischer Funktionen innerhalb der Codebasis. Diese Funktionen wiederum stellen spezifische Szenarien dar, um sicherzustellen, dass die API innerhalb der erwarteten Parameter funktioniert und dass Fehler gut behandelt werden, wenn die Ergebnisse außerhalb der erwarteten Parameter liegen. In der grundlegendsten Form sollte ein korrekter Response-Body die Darstellung der angeforderten Ressource unter Verwendung des angeforderten Medientyps oder mindestens einer gültigen HTTP-Fehlercodemeldung enthalten. Funktionstests sollten nicht nur den regulären Testfall einbeziehen, sondern auch Szenarien von Errata- und Edge-Cases in das Testverfahren implementieren. Ein Beispiel hierfür ist eine Anfrage mit falschen Attributen oder Attributwerten, auf die die API mit einem korrekten Fehlercode antworten sollte. Eine bei APIs besonders wichtige Unterkategorie der Funktionstests sind die Regressionstests, mit denen sichergestellt werden kann, dass bei internen Änderungen die Rückgaben der API weiterhin mit dem definierten Interface kompatibel sind, und somit Client-Anwendungen, die auf die API angewiesen sind, weiterhin wie bisher funktionieren.

\subsubsection{Last- und Stresstests}
Hierbei wird getestet, ob sich das System auch unter hoher oder höchster Belastung so verhält, wie es gefordert war \parencite[458]{ludewig2007software}. Diese Tests werden typischerweise nach Fertigstellung der Codebasis durchgeführt. REST APIs werden gegen den theoretischen regulären Datenverkehr, den die API im normalen, täglichen Gebrauch erwartet, getestet. Ein zweiter Belastungstest wird typischerweise mit dem theoretischen maximalen Traffic durchgeführt. Eventuell wird auch ein dritter Belastungstest mit Traffic, der über dem theoretischen Maximum liegt, durchgeführt.

\subsubsection{Sicherheitstests}
Sicherheitstests und Penetrationstests sind Teil des Sicherheitsauditprozesses. Bei Sicherheitstests wird zum Beispiel die Verwaltung von Benutzerrechten und die Validierung von Berechtigungsprüfungen für den Ressourcenzugriff getestet. Die Penetrationstests gehen einen Schritt weiter. Bei dieser Art von Test wird die API von jemandem angegriffen, der nur über begrenzte Kenntnisse der API selbst verfügt, um den Bedrohungsvektor aus einer externen Perspektive zu beurteilen.

\newpage

\subsection{Verwandte Arbeiten und Systeme}

Zweck dieses Unterkapitels ist die Darstellung von verwandten Arbeiten, welche sich ebenfalls mit der Automatisierung von Testing und Dokumentation von REST APIs beschäftigen. Daneben werden ebenfalls bereits existierende oder im Stadium der Entwicklung befindende Systeme einbezogen. Die betrachteten Systeme und Arbeiten werden mit der Zielsetzung des vorliegenden Projektes verglichen und Unterschiede und Gemeinsamkeiten hervorgehoben.

\subsubsection{Wissenschaftliche Arbeiten}

In einem Paper aus 2017 haben \textcite{ed2018automatic} ein System zur automatischen Generierung von Testcases auf Basis einer OpenAPI Spezifikation entworfen und als Prototyp implementiert. Es wird dafür zuerst die Spezifikation in ein OpenAPI Metamodell extrahiert, welches dann in mehreren Schritten mit Beispielwerten für Parameter und Attribute angereichert und anschließend in ein Testsuite Metamodell umgewandelt wird. Hieraus werden im letzten Schritt die entsprechenden Testcases generiert. Berücksichtigt werden hierbei nicht nur nominale sondern auch fehlerhafte Tests (also HTTP-Anfragen mit bewusst falschen Parameter- oder Attributwerten, mit denen auf entsprechende Validierungsfehler geprüft wird). Das System kommt der in diesem Projekt gesetzten Zielsetzung und dem dafür gewählten Vorgehen sehr nahe, es gibt jedoch einige Einschränkungen, über die sich dieses Projekt abgrenzen kann:
	\begin{itemize}
		\item Das System ist auf ein bestimmtes Spezifikationsformat (OpenAPI) beschränkt, und nicht mit anderen Formaten kompatibel. Es wurde zwar ein Metamodell erstellt, in welches die OpenAPI Spezifikation übertragen wird (und somit über entsprechende Erweiterungen eventuell auch andere Formate), bei der Erstellung dieses Modells wurden jedoch nur die Charakteristiken der OpenAPI Spezifikation berücksichtigt. Ein System welches mehrere Formate unterstützt, sollte bereits bei der Modellierung entsprechende Analysen und Abstraktionen anwenden.
		\item Der Prototyp ist lediglich als Eclipse-Plugin mit Generierung der Testcases in Java implementiert worden, und somit auf eine bestimmte Entwicklungsumgebung und Sprache (Java) beschränkt. Das Ziel dieses Projektes ist die Implementierung einer CLI-Applikation, welche so modular gestaltet ist, dass mehrere Input- und Output-Formate möglich sind. Aufgrund der Vorgaben im Unternehmen wird für die Implementierung des Prototyps als Input eine OpenAPI Spezifikation und als Output PHP gewählt.
	\end{itemize}
	
In einem Paper aus 2015 stellten \textcite{fertig2015model} einen modellbasierten Ansatz zur Generierung von Testcases vor. Hierzu wurde eine eigene Description Language erstellt, mit der die REST API beschrieben werden kann. Der Fokus liegt bei der Generierung von möglichst vielen Testcases; für jeden Endpunkt werden alle möglichen Kombinationen von Headern und Attributen abgebildet. Es werden zwar auch hier zum Teil fehlerhafte Tests generiert, mit dem Ziel zu überprüfen, ob der Server fehlerhafte Angaben korrekt abweist, jedoch sind diese im Vergleich zum oben vorgestellten System sehr beschränkt, und hängen stark davon ab, dass in der Spezifikation komplette Definitionen vorhanden sind. Ebenso scheinen optionale Query-Parameter nicht unterstützt zu werden. Die Implementierung des Prototyps wurde ebenfalls für Java entwickelt. Die größte Einschränkung ist jedoch die Verwendung einer eigenen DL, wodurch zusätzliche Einstiegshürden entstehen, und Tools für bestehende Spezifikationsformate nicht verwendet werden können.\\

\textcite{pinheiro2013model} verwenden einen ähnlichen Ansatz, basieren ihr Modell jedoch auf einer angepassten \emph{UML protocol state machine}, welche vom Tester manuell erstellt werden muss. Die Umsetzung fand hier jedoch nur im Ansatz statt, und wichtige Aspekte wie Ressourcenrepräsentationen oder optionale Parameter wurden nicht implementiert.\\

\textcite{chakrabarti2009test} entwickelten ein HTTP-Testing-Tool, das speziell für das Testen von RESTful-Webservices entwickelt wurde. Es basiert auf XML-Dateien, welche die erforderlichen Testfälle konfigurieren. In einem Pilotprojekt wurden Testfälle zur automatischen Generierung erstellt, die einer vollständigen Liste aller gültigen Kombinationen von Query-Parameter-Werten entsprechen. Daraus ergab sich eine Testsuite mit 42.767 Testfällen. Ein wesentlicher Nachteil des Systems ist das benötigte Vorwissen zur Konfiguration der Testcases mithilfe der zum Teil recht komplexen XML-Definitionen. \\

Ein etwas komplexerer Ansatz wurde von \textcite{arcuri2017restful} vorgestellt. Hier wird ebenfalls von einer OpenAPI bzw. Swagger Definition ausgegangen, und auf Basis dieser Tests generiert. Mithilfe von Bytecode-Manipulation bei JVM-Sprachen werden Informationen zu Code-Coverage und Verzweigungsabständen abgerufen, mit denen die optimalen Testcases ausgewählt werden. Abgesehen von der offensichtlichen Einschränkung auf JVM-Sprachen, prüft das System auch lediglich auf HTTP Statuscodes, und gleich die erhaltenen Antworten nicht mit Schemas o.ä. ab. \\

Ein Großteil der Arbeit in der Literatur konzentriert sich auf Black-Box-Tests von SOAP-Webservices, die mit WSDL (Web Services Description Language) beschrieben werden. Es wurden verschiedene Strategien vorgeschlagen, einige Beispiele sind \textcite{xu2005testing}, \textcite{bai2005wsdl}, \textcite{martin2006automated}, \textcite{ma2008wsdl} und \textcite{bartolini2009ws}. Diese werden jedoch aufgrund der unterschiedlichen Architektur und den damit geltenden Rahmenbedingungen nicht weiter betrachtet, und nur der Vollständigkeit halber aufgeführt.


\subsubsection{Vorhandene Softwarelösungen}

Zusätzlich zu den wissenschaftlichen Arbeiten gibt es auch viele bereits bestehende Softwarelösungen bzw. Tools, die mit API Spezifikationen arbeiten und bestimmte Aufgaben automatisieren können. Es sei erwähnt, dass zusätzlich zu den hier aufgelisteten Open Source Projekten auch noch einige kommerzielle, kostenpflichtige Lösungen verfügbar sind (z.B. Stoplight.io oder Readme.io), welche hier jedoch aufgrund der Vorgaben des Projektes seitens des Unternehmens nicht weiter betrachtet werden.\\

\textbf{Dokumentation}\\
\href{https://github.com/Rebilly/ReDoc}{ReDoc} ist ein Tool, mit dem auf Basis einer OpenAPI Spezifikation eine moderne, responsive HTML API Dokumentation generiert werden kann. Über ein Plugin werden ebenfalls Code-Beispiele in verschiedenen Sprachen unterstützt. Die Anpassung des Themes ist zwar grundsätzlich möglich, jedoch nicht dokumentiert. Nicht unterstützt werden ausführbare Beispielanfragen.

\href{https://swagger.io/tools/swagger-ui/}{Swagger UI} ist Teil des Swagger Frameworks, und ermöglicht ähnlich wie ReDoc das Generieren einer HTML API Dokumentation anhand einer OpenAPI Spezifikation. Anders als bei ReDoc sind hier ausführbare Beispielanfragen möglich, jedoch fehlen jegliche Anpassmöglichkeiten des Themes und Code-Beispiele.

\href{https://github.com/danielgtaylor/aglio}{Aglio} und \href{https://github.com/subosito/snowboard}{Snowboard} sind zwei Dokumentations-Tools für das API Blueprint Spezifikationsformat, welche jedoch schwer anpassbar sind.

Weitere nennenswerte Tools sind \href{https://github.com/sourcey/spectacle}{Spectacle} und \href{http://dapperdox.io/}{DapperDox}, mit ähnlichen Einschränkungen.\\

\textbf{Testing}\\
Zum Testen von APIs mithilfe einer API Spezifikation ist besonders \href{http://dredd.io/}{Dredd} zu erwähnen, ein sprachunabhängiges CLI-Tool zur Validierung von API-Spezifikationen gegen die Backend-Implementierung der API. Hierzu wird die API Spezifikation (im OpenAPI oder API Blueprint Format) eingelesen, und für jeden in der Beschreibung definierten Endpunkt eine Anfrage an den Server geschickt. Dafür werden alle definierten Parameter und Attribute mit ihren entsprechenden Beispielwerten benutzt. Die Antworten des Servers werden anschließend validiert; wird entweder ein nicht erfolgreicher Statuscode zurückgegeben, oder die Antwort entspricht nicht dem in der Spezifikation definierten Format schlägt der Test fehl. Dredd ist besonders hilfreich um zu testen, dass die API Spezifikation (und somit auch die Dokumentation) auf dem aktuellen Stand ist. Das eigentliche Testen der API Implementierung ist jedoch sehr eingeschränkt, da für jeden Endpunkt lediglich eine Anfrage ausgeführt wird (und somit eventuelle Query-Parameter nicht berücksichtigt werden), und ebenfalls keine fehlerhaften Anfragen abgeschickt werden.

\subsection{Abgrenzung von anderen Arbeiten}

Die vorliegende Arbeit hat das Ziel eine Gesamtlösung zur Vereinheitlichung und Optimierung des API-Entwicklungsprozesses zu entwerfen und zu implementieren. Zu Teilen dieses Konzeptes gibt es bereits, wie im vorherigen Abschnitt vorgestellt, einige Softwarelösungen. Im Folgenden werden kurz die Alleinstellungsmerkmale vorgestellt, die während der Analyse aufgefallen sind.\\

\textbf{Testing}\\
Zum Testing von REST APIs auf Basis einer API Spezifikation sind zwar einige wissenschaftliche Arbeiten zu finden, welche jedoch meist nur als Prototyp implementiert sind, und umfangreiche Anpassungen bzw. Erweiterungen benötigen würden um den Anforderungen gerecht zu werden. Einige Punkte, die von den untersuchten Systemen nicht abgedeckt werden, sind:
\begin{itemize}
	\item Generierung von Testcases in einer anderen Sprache als Java, in dem Fall dieses Projektes primär in PHP.
	\item Unterstützung von mehr als einem API-Spezifikationsformat. Die meisten Arbeiten basieren auf einer OpenAPI Spezifikation, verwendet werden in der Industrie jedoch auch andere (bspw. API Blueprint oder RAML). Hier müssten zusätzliche Analysen und Überlegungen durchgeführt werden, wie mehrere Formate in ein gemeinsames, internes Datenmodell überführt werden können, und welche Transformationen dafür eventuell nötig sind.
	\item Modularer Aufbau, zur einfachen Erweiterung. Insbesondere beim Generieren der Testfälle ist es sehr wahrscheinlich, dass langfristig verschiedene Output-Formate unterstützt werden müssen. Das entwickelte Tool sollte also so aufgebaut sein, dass mit geringem Aufwand Testcases für andere Sprachen oder Frameworks generiert werden können (durch Entwicklung einer Erweiterung welche leicht in das Tool eingebunden werden kann).
\end{itemize}
Von \emph{Dredd} kann sich das Projekt alleine durch die höhere Abdeckung der Testcases (optionale Query-Parameter und fehlerhafte Anfragen) abgrenzen.\\

\textbf{Dokumentation}\\
Zur Generierung von Dokumentation wurde eine Vielzahl von Tools gefunden, die bereits einen Großteil der Anforderungen abdecken. Dennoch wurde bei den Open-Source Projekten keine Lösung gefunden, welche eine API Dokumentation \emph{auf Basis mehrerer Spezifikationsformate generiert}, die responsiv, interaktiv (durch ausführbare Beispielanfragen) und leicht anpassbar ist, sowie automatisch generierte Code Snippets enthält.
